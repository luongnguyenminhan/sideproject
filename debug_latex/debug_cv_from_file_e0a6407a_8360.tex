\documentclass[letterpaper, 11pt]{article}

\usepackage[utf8]{inputenc}
\usepackage[T1]{fontenc}
\usepackage{geometry}
\geometry{left=1in, right=1in, top=0.75in, bottom=0.75in}
\usepackage{xcolor}
\usepackage{enumitem}
\usepackage{ragged2e}

% Color Definitions
\definecolor{primary}{HTML}{3498DB}
\definecolor{secondary}{HTML}{2980B9}
\definecolor{text}{HTML}{2C3E50}
\definecolor{background}{HTML}{FFFFFF}

% Custom Commands
\newcommand{\sectiontitle}[1]{\vspace{0.1in}\noindent{\Large\color{primary} #1}\vspace{0.05in}\hrulefill}
\newcommand{\subsectiontitle}[1]{\vspace{0.05in}\noindent{\large\color{secondary} #1}}
\newcommand{\entry}[4]{
    \vspace{0.02in}
    \noindent\textbf{#1}
    \hfill
    #2
    \newline
    \textit{#3}
    \newline
    #4
}

\newcommand{\skill}[1]{\texttt{#1}}

\begin{document}

\pagestyle{empty} % Suppress page numbers

% Header
\begin{center}
    {\Huge \color{primary} Luong Nguyen Minh An} \\
    \vspace{0.05in}
    \small
    \color{text}
    Email: luongnguyenminhan02052004@gmail.com | Phone: 0921 421 727 | LinkedIn: https://www.linkedin.com/in/luongnguyenminhan | Address: Thu Duc, Ho Chi Minh City
\end{center}

\sectiontitle{Summary}
Ứng viên Luong Nguyen Minh An đang tìm kiếm cơ hội trong lĩnh vực Công nghệ Thông tin, đặc biệt là các vị trí liên quan đến Trí tuệ Nhân tạo (AI) và Phát triển Web/API. Với kinh nghiệm thực tập tại FPT Telecom – Meobeo.ai ở vai trò AI Intern, ứng viên đã thể hiện khả năng vượt trội trong việc phát triển các giải pháp AI tự động hóa quy trình, bao gồm tạo ghi chú cuộc họp, xử lý âm thanh với Pyannote, nhận dạng giọng nói, và tích hợp lịch Google, mang lại hiệu quả giảm thời gian tìm kiếm thông tin tới 50\% và giảm nỗ lực ghi chú thủ công tới 60\%. Ứng viên còn tham gia vào các dự án cá nhân với vai trò AI Engineer, xây dựng chatbot bán hàng sử dụng kiến trúc RAG và LangChain, cũng như chatbot tài chính tích hợp xử lý chi phí và OCR. Về kỹ năng, ứng viên thành thạo Python, các nguyên tắc thiết kế hướng đối tượng, phát triển API với FastAPI, các framework AI như LangChain/LangGraph, và có kiến thức sâu về Machine Learning, Deep Learning, Computer Vision, NLP, cơ sở dữ liệu (MongoDB, MySQL, Qdrant) và DevOps (Docker, Kubernetes, CI/CD). Ứng viên còn sở hữu nhiều chứng chỉ chuyên ngành từ Coursera về Phát triển Phần mềm, Xử lý Ngôn ngữ Tự nhiên, Khoa học Dữ liệu, Phát triển Full Stack và Học sâu. Ứng viên đang theo học ngành Công nghệ Thông tin tại Đại học FPT với GPA 8.6/10 và dự kiến tốt nghiệp tháng 7/2025.

\sectiontitle{Experience}
\entry{FPT Telecom – Meobeo.ai}{Ho Chi Minh City}{AI Intern}{2024-09-01 – Present}
\begin{itemize}[leftmargin=0.2in]
    \item Engineered a modular, flow-based agent in Python to fully automate meeting-note generation via advanced prompt engineering and context-aware summarization, enabling real-time note delivery.
    \item Orchestrated end-to-end integration of speech-to-text and Pyannote for speaker diarization, producing enriched, timestamped transcripts with keyword extraction—reducing information retrieval time by 50\%.
    \item Architected a speaker-identification module using voice-embedding techniques and clustering algorithms to recognize and remember individual speakers across sessions uniquely.
    \item Developed synchronization logic via RESTful connectors to Google Calendar, automatically aligning key insights, follow-up tasks, and action items with users’ calendars and sending contextual email summaries.
    \item Secured session management and data protection with JWT-based authentication, ensuring GDPR-compliant handling of sensitive meeting data.
    \item Deployed containerized microservices using Docker Compose and Kubernetes, with automated CI/CD pipelines in GitHub Actions—achieving zero-downtime rollouts and horizontal scaling to support 200+ concurrent users.
    \item Accelerated organizational productivity by cutting manual note-taking effort by 60\% and improving meeting follow-through rates by 40\%
\end{itemize}

\sectiontitle{Education}
\entry{FPT University}{}{Information Technology}{Expected Graduation: 2025-07-01, GPA: 8.6}

\sectiontitle{Projects}
\entry{AI-driven Sales Consultant Chatbot}{}{}{Technologies: FastAPI, MongoDB, MySQL, Qdrant VectorDB, LangChain, LangGraph, ReAct Agent}
\begin{itemize}[leftmargin=0.2in]
    \item Engineered a FastAPI-based chatbot API, integrating MongoDB, MySQL, and Qdrant VectorDB for seamless data storage.
    \item Implemented Retrieval-Augmented Generation (RAG) architecture to ensure highly relevant information retrieval.
    \item Leveraged LangChain and LangGraph frameworks to integrate large language models for Python-based solutions.
    \item Orchestrated a ReAct Agent-based sales consultant chatbot to address complex customer inquiries dynamically.
    \item Applied SOLID principles and clean-code practices to maintain a scalable, maintainable backend.
    \item Authored comprehensive API documentation and in-line code comments to facilitate future development.
    \item Structured the project repository following industry best practices for FastAPI applications.
\end{itemize}

\entry{Financial Chatbot - Third-Party}{}{}{Technologies: LangChain, LangGraph, LLM APIs, RAG}
\begin{itemize}[leftmargin=0.2in]
    \item Developed a multi-turn chatbot using the LangChain LangGraph framework and LLM APIs to deliver intelligent financial guidance.
    \item Designed and implemented a RAG system to provide accurate, context-aware financial responses.
    \item Managed user expense records with full CRUD operations, generating personalized financial insights on demand.
    \item Built an intuitive chat interface enabling users to rapidly input and categorize income and expenses.
    \item Optimized mobile OCR algorithms to extract expense details from transaction screenshots with high accuracy.
    \item Devised preprocessing routines to redact sensitive bank details before external API calls, bolstering data security.
    \item Executed model quantization techniques, reducing memory footprint and boosting recognition speed by over 30\% on mobile devices.
\end{itemize}

\entry{WSTE: Whitebox Style Transfer with Example-based Learning}{}{}{Technologies: VGG network, STROTSS algorithm}
\begin{itemize}[leftmargin=0.2in]
    \item Conducted a comprehensive literature review to define project methodology and benchmarks.
    \item Analyzed VGG network architecture within the STROTSS algorithm to drive image style transfer innovations.
    \item Integrated image segmentation techniques to enable multi-style fusion within single images.
\end{itemize}

\sectiontitle{Skills}
\begin{itemize}[label={}, itemsep=0.1em]
    \item \textbf{Programming Languages:} \skill{Python}, \skill{R}
    \item \textbf{Programming Principles:} \skill{Object-Oriented Design}, \skill{SOLID Principles}, \skill{Clean Code}
    \item \textbf{AI Techniques:} \skill{Prompt Engineering}, \skill{ReAct Agent}, \skill{Speaker Diarization}, \skill{Speaker Identification}
    \item \textbf{Web Frameworks:} \skill{FastAPI}
    \item \textbf{API Design:} \skill{RESTful API Design}
    \item \textbf{Frameworks:} \skill{Flow-Based Agent Frameworks}, \skill{LangChain}, \skill{LangGraph}
    \item \textbf{Tools:} \skill{API Documentation}, \skill{PowerBI}, \skill{MS Excel}, \skill{Docker}, \skill{Docker Compose}, \skill{Kubernetes}, \skill{GitHub Actions}, \skill{Git}
    \item \textbf{Cloud Platforms:} \skill{EKS}, \skill{AWS}, \skill{Azure}
    \item \textbf{DevOps Practices:} \skill{CI/CD}
    \item \textbf{Fields:} \skill{Machine Learning}, \skill{Deep Learning}, \skill{Computer Vision}, \skill{Natural Language Processing}, \skill{Audio Processing}
    \item \textbf{Libraries:} \skill{Pyannote}
    \item \textbf{Databases:} \skill{MongoDB}, \skill{MySQL}, \skill{SQLite}, \skill{MS SQL Server}, \skill{Qdrant}, \skill{VectorDB}
    \item \textbf{Skills:} \skill{Data Analysis}, \skill{Data Visualization}
    \item \textbf{Soft Skills:} \skill{Critical Thinking}, \skill{Problem Solving}, \skill{Collaboration}, \skill{Communication}, \skill{Research}, \skill{Technical Writing}, \skill{Presentation}
\end{itemize}

\sectiontitle{Certificates}
\begin{itemize}[leftmargin=0.2in]
    \item Software Development Lifecycle Specialization - Coursera
    \item Natural Language Processing Specialization - Coursera
    \item Applied Data Science with Python Specialization - Coursera
    \item IBM Full Stack Software Developer Specialization - Coursera
    \item Deep Learning Specialization - Coursera
    \item Big Data Specialization - Coursera
\end{itemize}

\end{document}